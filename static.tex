\documentclass[-no math,oneside]{book}
\usepackage{xeCJK}
\usepackage{indentfirst}
\usepackage{amsmath}
%xeCJK宏包使用了和fontspec宏包非常类似的语法设置中中文字体
%fontspec宏包提供了几个设置全局字体的命令,
%setmainfont{<font name>}[<font features>]
%setsansfont{<font name>}[<font features>]
%setmonofont[<font name]}[<font features>]
%这样的格式是新版宏包推荐使用的格式。
%在上面的格式中,<font name>使用字体的文件名或者字体的英文名称。
%<font features>用来手动配置对应的粗体或斜体。
%由于中文字体少有对应的粗体或斜体,<font features>
%里多用其他字体来配置,
\setCJKmainfont{宋体}[BoldFont={黑体}, ItalicFont={楷体}]
\setCJKsansfont{黑体}
\setCJKmonofont{黑体}
\setlength{\parindent}{2em}
\fontsize{12pt}{1.2}
\begin{document}
\chapter{梳理统计的基本概念}
\section{引言}
概率论与梳理统计都是研究随机现象数量规律性的学科。在学习了概率论之后为什么还要学习数理统计,概率论与数理统计之间的区别又在什么地方呢?在概率论部分的学习中,我们所遇到的随机事件,其发生的概率有的假定是已知的,有的可以由已知条件或其他事件的已知概率推算出来;我们所遇到的随机变量,其分布函数也常假定是已知的。例如,某种电子元件的废品率
$p$
已知:
$p=0.01$
,现人去10个这种电子元件,求其中恰有1个废品的概率。
由于随机性的影响无所不在,所以数理统计的应用十分广泛,不仅在工农业生产、工程技术、自然科学等领域有着广发的应用,在医学卫生、社会生活以及经济等领域也有越来越广泛的应用。许多统计方法的应用都涉及大量的计算。近年来,由于不断升级换代的电子计算机这一有力工具的介入,使得某些统计方法计算是大的障碍已不复存在,这也促使数理统计获得长足的发展。数理统计已成最活跃的数学分支学科之一。

\section{总体与样本} 

数理统计学中把全体研究对象所组成的集合称为总体。总体中的每个元素称为个体。这里所说的研究对象详细地说应该是,那个对象的某项(或某几项)数量指标。例如前面所说的某厂生产的一大批灯管,因为我们关心的是这批灯管的寿命,而不是其电阻、长度等其他特性,所以这批灯管的寿命的全体是一个总体,其中的每个灯管的寿命是一个个体。又例如,在研究某大学学生的体重状况时,我们对于这些学生的身高、学习成绩、籍贯等都不感兴趣。该校所有学生体重的全体是一个总体,而每个大学生的体重就是一个个体。总体按照其所包含的个体的总数是有限还是无限,分为有限总体和无限总体。在实际问题中总体一般都是有限的。如前面所说的某厂生产的一大批灯管,无论这批灯管的数量有多大,其总数总是有限的。由这批灯管寿命的全体构成的总体就是一个有限总体。为了研究问题的方便,我们会把这批灯管寿命的取值范围扩大到在相同条件下所有可能生产出来的灯管的寿命。这时我们所研究的总体就成了一个无限总体。

在获得每个个体之前,我们知道它取值的范围,却无法预知其具体值。所以总体中每个个体的出现都是一次随机实验的结果,也就是说,个体是某个随机变量
$X$
的一个观察值。对总体的研究对手对相应的随机变量
$X$
的研究。以后我们说到“总体
$X$
”,这个
$X$
就是与总体相应的随机变量。每个总体
$X$
的概率分别在客观上是存在的,但是我们不知道,至少是部分未知。以后我们总假定总体
$X$
的分布函数
$F(x)$
或者数学形式已知,但其中的参数未知;或者
$F(x)$
的数学形式也不知道。如果总体
$X$
的分布函数
$F(x)$
全部已知,那就没有统计工作可做了;这时关于
$X$
概率性质的研究便是概率论要做的事了。

在前面有关灯管寿命的例子中,如果根据历史知识或经验,在做统计推断之前就可假定灯管寿命
$T$
服从指数分布,其分布函数为
\[
F(t) = 
\begin{cases}
1 - e ^{-\gamma t} , & t> 0, \\
0 , & t<=0,
\end{cases}
\]
但其中的参数
$\gamma$
未知。这时称总体
$T$
属于指数分布族,也就是说
$T$
是指数分布族的一员。像这种有分布函数的形式确定,但带有几个参数的分布构成的分布族成为参数分布族。参数分布族常记为
$\{F(x; \theta), \theta \in \Theta\}$
,其中
$\theta$
表示参数,
$\Theta$
表示参数可能取值的范围,称为参数空间。

有这样一类常见的参数分布族
$\{F(), - \infty < a < + \infty, b > 0 \}$,其中
$a$
成为位置参数
$b$
成为尺度参数。如果该分布族的密度函数存在,且用密度函数来描述分布族,则该分布族可改写为。例如,正太分布族可改写为。

在有些情况下,我们对总体
$X$
知之甚少,以至于不能对
$X$
的具体分布形式作任何假定。只能作出
$X$
是连续型随机变量或
$X$
只能取正值等这样一般性的假定。这种不能用有限个实参数去刻划的分布族称为非参数分布族。由于参数分布族与非参数分布族的差异较大,导致在研究方法上差别很大。从参数分布族出发所得到的统计方法称为参数统计方法。而从非参数分布族出发所得到的统计方法则称为非参数统计方法。

为了研究总体的统计特性,必须从总体中任取一部分个体进行观察。总体中部分个体所组成的集合称为样本。样本中的每个个体称为样品。样本的个数称为样本容量。任取个体的过程称为抽样。所谓抽样,或任取个体,实际上就是进行随机试验,并观察所关心的数量指标。通过观察我们得到一组容量为
$n$
的样本
$(x1,x2,...,xn)$
。由于那个
$n$
个个体被去除进行观察或试验是随机的,因此每一xi均可看做随机变量Xi的观察值。我们也称
$(X1,X2,...,Xn)$
是容量为
$n$
的样本。为了细分,有时我们把
$(X1,X2,...,Xn)$
称为样本,把
$(X1,X2,...,Xn)$
的观察值
$(x1,x2,...xn)$
称为样本观察值或样本值。总之,我们是以两种观点来看待样本的,这就是样本的两重性。

抽样按其抽取个体方法的不同可分为两种:放回抽样和不放回抽样。如果抽取第一个个体并进行观察之后就放回,然后在抽取第二个个体,观察后让放回……直到取出第
$n$
个个体。这种抽样方法称为放回抽样。如果每取出一个观察后不在放回,直到取出第
$n$
个个体为止,或者一次性取出
$n$
个个体,这张抽样方法称为不放回抽样。对于无限总体,由于抽取的个体是否放回不改变总体的成分,因此这两种抽样方法的效果是一样的。对于有限总体,由若采取不放回凑杨,没抽取一个个体,总体的成分就要改变。但在世纪应用中,即使总体中个体数
$N$
有限,只有
$N$
相对于容量
$n$
很大(一般要求
$n/N<=0.1$
),仍采取不放回抽样的方法,并近似地认为抽取样本后总体的成分不变。

样本
$(X1,X,...,Xn)$
中
$Xi$
的观察值
$xi$
是对总体
$X$
进行第
$i$
次观察或试验的结果,所以每个
$Xi$
都与
$X$
有相同的分布。一般还假定这些试验是相互独立的,也就是说各次抽取个体获得的结果彼此互不影响。这样的到的样本称为简单随机样本。今后如果不做特殊说明,我们所说的样本都理解为简单随机样本,并简称样本。

1.3 统计量

一、统计量的定义

来自总体X的样本(X1,X2,...,Xn)含有总体的各种信息,因此样本是很宝贵的。我们作为统计推断的依据就是样本。但是如果不对样本进行“加工”、“提炼”,总体的各种信息仍分散在样本的每个样品中,称为一堆杂乱无章的数。为了充分利用样本所包含的关于总体的各种信息,必须把分散在各个样品中的信息集中起来。集中的方法就是对不同的问题构造各种各样的样本的函数,这种样本的函数就称为统计量。它的定义如下。
定义1.3.1
设(X1,X2,...X3)是来自总体X的样本,假如样本的实值函数g(X1,X2,...,Xn)不依赖任何未知量,则称g(X1,X2,...,Xn)为统计量。
定义1.3.2
设(X1,X2,...X3)是来自总体X的样本,则称

以上这些统计量成为样本矩。样本矩是最常用的统计量。记,称为修正后的样本方差。至于我们为什么不把称为样本方差,而把称为样本方差,其理由将在2.2中说明。

样本均值

二、次序统计量

定义1.3.4
定理1.3.1
定理1.3.2
定理1.3.3

三、样本中位数和极差


\end{document}